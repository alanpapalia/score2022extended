\section{Related Work}

The current state-of-the-art formulation of RA-SLAM is through nonlinear
least-squares (NLS) optimization based upon MAP inference
\cite{dellaert2017factor}. This work specifically focuses on
\textit{initialization} for this MAP formulation. Though we focus on MAP
estimation, there are many important previous formulations of RA-SLAM using:
extended Kalman filters \cite{Newman03icra, menegatti09icra,
Djugash09iros,djugash09springer}, particle filters \cite{gonzalez09ras,
blanco08icra}, mixture models \cite{blanco08iros}, and NLS optimization
\cite{boroson20iros, funabiki21ral, herranz14icra}.

We first survey previous initialization strategies for RA-SLAM problems and then
expand the scope to cover initialization strategies for general pose-graph SLAM
problems. As our initialization approach is a convex relaxation of the RA-SLAM
problem, we discuss related convex relaxations in robotic state estimation.
Finally, as the key challenge that differentiates RA-SLAM from pose-graph SLAM
is the nature of range measurements, we discuss the closely related field of
sensor network localization (SNL) and convex relaxations developed specifically
for SNL problems.

\subsection{Initialization in RA-SLAM}

Notable work in the single-robot RA-SLAM case used spectral graph clustering to
initially estimate beacon locations \cite{Olson06joe}. Similarly,
\cite{boots13icml} used spectral decomposition to jointly estimate robot poses
and beacon locations. However, these approaches \cite{Olson06joe,boots13icml} do
not account for sensor noise models and do not readily extend to multi-robot
scenarios. Other initializations for multi-robot RA-SLAM used coordinated
movement patterns to find the (single) relative transform between robots
\cite{guo2017ijmav,li20arxiv}.

Finally, a series of works developed increasingly sophisticated methods to
compute relative transforms between two robots using range and odometry
measurements
\cite{zhou08tro,trawny07iros,trawny10rss,trawny10tro,jiang20itaes,li20iros,Li2022ral}.
While these approaches represent great progress, they are only capable of
solving for a single inter-robot relative transform and rely on (noisy)
odometric pose composition to initialize all other poses. In fact,
\cite{jiang20itaes,li20iros,Li2022ral} utilize convex relaxation to solve the
relative transformation problem.

\subsection{Initialization and Convex Relaxation in State Estimation}

In pose-graph SLAM many existing initializations solve approximations of the MAP
problem. \cite{carlone15aicra} leveraged the relationships between rotation and
translation measurements in SLAM and explored the use of several rotation
averaging algorithms
\cite{martinec07cvpr,govindu01cvpr,fredriksson13lnsc,tron14tac} to obtain
initial estimates. The same relationships were used to approximate the SLAM
problem as two sequential linear estimation problems \cite{carlone14ijrr}.
However, these approaches do not consider range measurements and thus do not
extend to RA-SLAM problems.

Except for \cite{tron14tac}, these initialization procedures
represent convex relaxations of the pose-graph SLAM problem. Subsequent works in
pose-graph SLAM developed convex relaxations which obtained exact solutions to
the non-convex MAP problem \cite{rosen19ijrr,
carlone15iros,briales17ral,tron15rssworkshop, fan20tro }. Other works developed
exact convex relaxations for the problems of extrinsic calibration
\cite{giamou19ral}, two-view relative pose estimation
\cite{briales18cvpr,garcia-salguero21ivc}, and spline-based trajectory
estimation from range measurements with known beacon locations
\cite{pacholska20ral}. Except for \cite{pacholska20ral}, these convex
relaxations also do not extend to range measurements. However,
\cite{pacholska20ral} required known beacon locations, allowed only for range
measurements, and only considered single-robot localization. SCORE allows for
multi-robot RA-SLAM, and combines measurements of rigid transformations and
ranges without necessitating any information \textit{a priori}.

A number of convex relaxations exist for sensor network localization (SNL). As
SNL centers around point-to-point distance measurements, these works are closely
related to SCORE. Previous works  developed and analyzed semidefinite
program (SDP) \cite{biswas06tase,
so07mathematicalprogramming, shamsi13dgo} and SOCP \cite{tseng07siam,naddafzadeh-shirazi14twc,doherty01infocom,
srirangarajan08twc}
relaxations of the SNL problem. Whereas these relaxations only
consider range measurements and can only estimate the Euclidean position of
points, SCORE allows for rigid transformation measurements and estimation of
poses. Additionally, as SCORE is a SOCP, it is substantially more scalable than
these SDP relaxations.

\subsection{Novelty of Our Approach}

SCORE is similar to \cite{martinec07cvpr} in its relaxation of rotations and to
\cite{naddafzadeh-shirazi14twc} in its relaxation of range measurements.
However, it differs from these works in that it jointly considers both relative
pose and range measurements. Notably, SCORE generalizes the
chordal-initialization of \cite{martinec07cvpr} and the SOCP relaxation of
\cite{naddafzadeh-shirazi14twc}.

Additionally, there is much commonality to
\cite{jiang20itaes,li20iros,Li2022ral} in that the convex relaxation of SCORE
considers both range and relative transformation measurements. Unlike these
works, SCORE generalizes to arbitrary dimensions (e.g. 2- or 3-D) and
multi-robot cases. Furthermore, SCORE differs in that the convex relaxation is
derived from the full MAP problem, and thus utilizes all measurements and
jointly solves for an initialization of all RA-SLAM variables.

