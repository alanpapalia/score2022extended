\section{Discussion}

In summary, the results shown in
\cref{sec:experiments} demonstrated
that SCORE works in real-world settings and outperforms state-of-the-art
initialization techniques for a wide range of RA-SLAM problems. We emphasize
the difficulty of the problems presented and that SCORE is capable
of providing quality initializations for multi-robot RA-SLAM problems with only
odometry and inter-robot range measurements.

Furthermore, in evaluating SCORE's performance on these multi-robot experiments
we emphasize that Odom-R is the only other initialization technique which
assumes the same amount of prior information as SCORE, i.e., no information
beyond the robot's measurements. In comparing SCORE to Odom-R we observe that
the SCORE initialization typically obtains an average APE roughly two orders of
magnitude less than that obtained by Odom-R initialization. In fact, SCORE
appears to obtain more accurate trajectory estimates than those generated by
Odom-P, which assumes knowledge of each robot's first pose. These
observations suggest that SCORE is a more effective initialization strategy than
general odometry-based techniques (the current state-of-the-art initialization
for these problems) even when each robot's first pose is
available.

In several instances the SCORE initialization was a qualitatively poor solution,
but local refinement of the SCORE initialization resulted in high-quality final
estimates. These seemingly low-quality SCORE initializations had notably
incorrect translations (e.g., the estimated distances traveled were much less
than the ground truth values). However, these same initializations appeared to
have high-quality orientation estimates and relative arrangement of translation
variables (robot and landmark positions). As high-quality solutions were
obtained by refining these initializations, these experiments reveal similar
phenomena to results in pose-graph SLAM \cite{carlone15aicra}, which suggested
that the quality of the orientation initialization played an outsized role on
the final, locally refined estimate.

Finally, based on results in \cref{sec:runtime-evaluation} we found that, while
not fast enough for real-time state-estimation, SCORE can scale to meaningful
problem sizes and run fast enough to be deployed in real-world robotics
applications. We note that, similarly to most SLAM problems, the SCORE problem
is inherently ill-conditioned. As a result, we used more conservative
optimization parameters to ensure that SCORE could be reliably solved for a
wide-range of problems. Thus, it is possible that substantial efficiency gains
could be found by parameter tuning for specific applications as well as by
exploration of other techniques to improve numerical stability.


\subsection{When Does SCORE Return Poor Initializations?}
\label{sec:limitations}

\ScoreDeterminantPlot

Though SCORE generally performed well as an initialization technique, we
observed several multi-robot RA-SLAM instances where SCORE produced poor
initializations. These poor initializations were due to the cost function
tending towards a trivial solution of all zeros (i.e., $\rot_i = 0_{d \times d},
\tran_i = \dist_i = 0_{d}$). While the constraint $\rot_0 = I_{d}$ should
prevent this, in practice certain scenarios demonstrate the
remaining variables rapidly decaying to zero. This phenomenon is unsurprising,
as it is found in similar SLAM relaxations \cite{Rosen15icra,carlone15aicra} and
the zeros elements indeed minimizes the problem cost. Importantly, we
note that these failures can be readily detected by observing the determinants
of the estimated rotations, which will quickly decay to less than
$\expnumber{5}{-2}$ (see \cref{fig:rotation-determinants}).
As a result, failures can be identified and alternative initializations used in
this case.

We find that the experiments with failed (poor) initializations resemble the
conditions identified by \cite{tseng07siam} for the SNL SOCP derived.  Namely,
the SOCP relaxation returns good solutions when the robot trajectories had
well-constrained poses (via range measurements) within the convex hull of robot
poses connected via odometry to $\rot_0$ (i.e., the trajectory of the robot with
constrained first rotation). Though we currently lack formal analysis, empirical
results suggest close connections to the rigidity theory established in
\cite{tseng07siam}.
