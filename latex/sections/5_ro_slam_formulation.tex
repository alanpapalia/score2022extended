\section{Range-Aided SLAM Formulation}

In this section we model the RA-SLAM problem and derive its
corresponding MAP estimation problem.

\subsection{Inference Over a Graph}

We formulate RA-SLAM as inference over a directed graph, with nodes as variables
(i.e. robot poses and landmark positions) and the edges representing
measurements. Without loss of generality, we assume directionality in the edges.
The edge-set of all measurements of relative rigid transformations is denoted
$\TransformEdges$, where $\dedge \in \TransformEdges$ represents a measured
rigid transformation from node $i$ to node $j$. Similarly, the edge-set of all
range measurements is denoted $\RangeEdges$ and each $\dedge \in \RangeEdges$
represents a measured distance from node $i$ to node $j$.

\subsection{Measurement Noise Models}
\label{sec:noise-models}

For relative rigid transformation measurements we follow the formulation of
\cite{rosen19ijrr} in the noise model we assume over our
measurements. We denote the true translation and orientation of node $i$ as
$\ttran_i \in \dvec$ and $\trot_i \in \SOd$, with the true relative rotation from node $i$ to node
$j$ as $\trot_{ij} \triangleq \trot_{i}^\top \trot_{j}$. Similarly, we denote
the true relative translations as $\ttran_{ij} \triangleq \trot_{i}^\top (
        \ttran_{j} - \ttran_{i} )$. We indicate noisy measurements with a tilde (e.g.,
$\nrot$):
\begin{equation}
    \begin{aligned}
        \ntran_{i j} & =\ttran_{i j}+\ptran_{i j}, & \ptran_{i j} & \sim \mathcal{N}\left(0, \tau_{i j}^{-1} I_{d}\right), \\
        \nrot_{i j}  & =\trot_{i j} \prot_{i j},   & \prot_{i j}  & \sim \langevin \left(I_{d}, \kappa_{i j}\right).
    \end{aligned}
\end{equation}

We assume the following generative Gaussian model for range measurements:
\begin{equation}
    \label{eq:range-measurement-model}
    \begin{aligned}
        \ndist_{i j}   & = \lVert \ttran_i - \ttran_j \rVert_2
        +\pdist_{i j}, & \pdist_{i j}                          & \sim \mathcal{N}\left(0, \sigma_{i j}^2 \right). \\
    \end{aligned}
\end{equation}

\subsection{MAP Estimation for RA-SLAM}

We write the MAP problem corresponding to the sensor models of
\cref{sec:noise-models}. The MAP problem is as follows, where $\tran_i \in
    \dvec$ and $\rot_i \in \SOd$ denote the variables corresponding to the estimated
translation and rotation of node $i$. Bolded symbols indicate groupings (e.g.,
$\textbf{\tran} \triangleq \{\tran_i~\forall~i=1,\ldots,n \}$):
\begin{equation}
    \label{eq:general-MAP}
    \begin{array}{r}
        \max\limits_{ \bf\tran , \bf\rot } ~ \prob ( \bf \ntran , \bf \nrot, \bf \ndist \mid \bf \tran , \bf \rot).
    \end{array}
\end{equation}
This general MAP problem of \cref{eq:general-MAP} is equivalently solved by
minimizing its negative log-likelihood \cite{dellaert2017factor}. Given the
models of \cref{sec:noise-models}, the MAP estimate of the RA-SLAM
problem can be expressed as the following NLS problem:
\begin{equation}
    \label{eq:ra-slam-nls}
    \begin{aligned}
        \min\limits_{\substack{ \tran_i \in \dvec                                                                                    \\ \rot_i \in \SOd}} ~~
          & \sum\limits_{\dedge \in \TransformEdges} \kappa_{ij} \lVert \rot_j - \rot_i \nrot_{ij} \rVert_F^2                        \\
        + & \sum\limits_{\dedge \in \TransformEdges} \tau_{ij} \left \lVert \tran_j - \tran_i - \rot_i \ntran_{ij} \right \rVert_2^2 \\
        + & \sum\limits_{\dedge \in \RangeEdges} \frac{1}{\sigma^2_{ij}} ( \lVert \tran_i - \tran_j \rVert_2 - \ndist_{ij})^2        ~ .
    \end{aligned}
\end{equation}
where $\| \cdot \|_F$ denotes the matrix
Frobenius norm:

In \cref{eq:ra-slam-nls} the first two summands correspond to relative
transformation measurements whereas the third summand corresponds to
range measurements. Derivation of the cost terms in the first two summands can
be found in \cite{rosen19ijrr}. The cost terms in the third summand follow
immediately from the negative log-likelihood of the model in
\cref{eq:range-measurement-model}.

Notice that there are two distinct sources of non-convexity in \cref{eq:ra-slam-nls}. The rotation variables are members of the special orthogonal group,
i.e., $\rot_i \in \SOd$, which is a non-convex set. Additionally, the cost terms
due to range measurements are non-convex due to the $ \lVert \tran_i - \tran_j
    \rVert_2 $ component.
