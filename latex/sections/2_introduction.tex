\section{Introduction}

Range-aided simultaneous localization and mapping (RA-SLAM) is a key robotic
task, with applications in planetary \cite{boroson20iros}, subterranean
\cite{funabiki21ral}, and sub-sea \cite{Newman03icra,Bahr06iser,Bahr12iros}
environments. RA-SLAM combines range measurements (e.g., distances between
acoustic beacons) with measurements of relative rigid transformations (e.g.,
odometry) to estimate a set of robot poses and landmark positions.

RA-SLAM differs from pose-graph simultaneous localization and mapping (SLAM) in
that the sensing models of range measurements induce substantial difficulties
for state-estimation. However, ranging is a valuable sensor modality and
further developments in RA-SLAM could substantially advance the state of robot
navigation.

The state-of-the-art formulation of SLAM problems is as \textit{maximum a
posteriori} (MAP) inference, which takes the form of an optimization problem.
However, because robot orientations are a non-convex set, the MAP problem is
non-convex. Additionally, in RA-SLAM, the measurement models of range sensing
introduce further non-convexities. As a result, standard approaches to solving
these MAP problems use local-search techniques and can only guarantee locally
optimal solutions. Thus, good initializations to the MAP problem are key to
obtaining quality RA-SLAM solutions.

\TitleFigure

This work approaches initialization for the non-convex MAP estimation problem
through \textit{convex relaxation} \cite{boyd04book}, the construction of a
convex optimization problem which attempts to approximate the original problem.
Convex problems can be efficiently solved to global optimality, and many convex
relaxations have shown success as initialization strategies in related robotic
problems
\cite{carlone15icra,carlone15iros,giamou19ral,so07mathematicalprogramming}.

This work presents SCORE as a novel methodology for initializing RA-SLAM
problems. SCORE applies to 2D and 3D scenarios with arbitrary numbers of robots
and landmarks. As SCORE is a SOCP, it is easily implemented in existing
convex optimization libraries, and scales gracefully to large problems. We
summarize our contributions as follows:

\begin{itemize}
      \item A novel, convex-relaxation approach to initializing RA-SLAM
            problems.
      \item The first QCQP formulation of RA-SLAM, connecting
            RA-SLAM to a broader body of work
            \cite{rosen19ijrr,carlone15iros,giamou19ral,so07mathematicalprogramming}.
      \item Our implementation of SCORE is open-sourced\footnote{\url{\RepoURL}}.
\end{itemize}

