\section{SCORE: Second Order Conic Relaxation}
\label{sec:ra-slam-score}
In this section we reformulate the MAP problem of Prob. \ref{prob:ra-slam-nls} as a
QCQP. We then show how to construct SCORE by relaxing this QCQP to a SOCP.

\subsection{RA-SLAM as a QCQP}
\label{sec:ra-slam-qcqp}
 We perform two changes to convert the non-convex NLS problem of
 Prob. \ref{prob:ra-slam-nls} problem to the QCQP of Prob. \ref{prob:ra-slam-qcqp}. First, we
 rewrite the $\SOd$ constraint as the orthonormality constraint and remove the
 associated determinant constraint.  Secondly, we introduce the auxiliary
 variables, $\dist_{ij} \in \RnonNeg$ to rewrite the range-measurement cost
 terms as quadratics.  In this formulation the cost terms are quadratic and
 convex with respect to the problem variables and all constraints are quadratic
 equality constraints. Thus, Prob. \ref{prob:ra-slam-qcqp} is a QCQP.

Previous work \cite{carlone15iros,briales17ral,tron15rssworkshop} demonstrated
that $\det(\rot_i) = 1$ can be written as a set of quadratic equalities and that
the contribution of this determinant constraint is negligible, justifying its
removal. Except for the determinant constraint, this QCQP is an exact
reformulation of Prob. \ref{prob:ra-slam-nls}:
\RaSlamQcqp

\subsection{Relaxing Problem \ref{prob:ra-slam-qcqp} to a SOCP}
\label{sec:ra-slam-socp}

We will now introduce the primary contribution of our paper, a \ScoreFull~(SCORE). As
previously mentioned, SCORE naturally arises as a relaxation of the QCQP in
Prob. \ref{prob:ra-slam-qcqp}. We emphasize that Prob. \ref{prob:ra-slam-qcqp} is not solved
in our approach, as it is not computationally practical to do so.
Prob. \ref{prob:ra-slam-qcqp} is important as an intermediate representation to derive
SCORE.

We relax the orthonormality constraints by removing them entirely, eliminating
that specific source of non-convexity. To eliminate the non-convexity from the
quadratic equality constraint on our auxiliary variables ($\dist_{ij}$) we first
recall that $\dist_{ij} \geq 0$, so we can drop the squares on each side of the
inequality without modifying the feasible set. From here, we can relax the
equality constraint to the convex inequality constraint $ \dist_{ij} \geq \lVert
    \tran_i - \tran_j \rVert_2$. Importantly, this specific constraint relating the
auxiliary variables and position variables is a (convex) second-order cone
constraint \cite{boyd04book,alizadeh03mathematicalprogramming}. This relaxed
problem takes the following form, where $\CostFunction$ is the same cost
function as in Prob. \ref{prob:ra-slam-qcqp}.
\begin{equation}
    \label{prob:ra-slam-socp}
    \begin{aligned}
         \min\limits_{\bf \tran, \bf \rot, \bf \dist} ~~ & \CostFunction \\
         \st & \dist_{ij} \geq \lVert \tran_i - \tran_j \rVert_2,~\forall \dedge \in \RangeEdges \\
          & \rot_0 = I_d
    \end{aligned}
\end{equation}
With this previously mentioned second-order cone constraint on $\dist_{ij}$, it follows that
Prob. \ref{prob:ra-slam-socp} is a SOCP, i.e., it is the minimization of quadratic cost
functions over the intersection of an affine subspace with second-order
cones. This is a known form of SOCPs \cite{alizadeh03mathematicalprogramming},
and thus existing solvers (e.g., \cite{gurobi}) can efficiently solve
Prob. \ref{prob:ra-slam-socp} to global optimality. Similarly to \cite{martinec07cvpr},
the first rotation ($\rot_0$) is constrained to prevent a trivial solution of
all zeros.
