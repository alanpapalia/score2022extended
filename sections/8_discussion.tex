\section{Discussion}

In summary, the results shown in
\cref{sec:experiments} demonstrated
that SCORE works in real-world settings and outperforms state-of-the-art
initialization techniques for a wide range of RA-SLAM problems. We emphasize
the difficulty of the multi-robot problems presented and that SCORE is capable
of providing quality initializations for multi-robot RA-SLAM problems with only
odometry and inter-robot range measurements.

\MultiRobotAPEBoxPlot

Furthermore, in evaluating SCORE's performance on these multi-robot experiments
we emphasize that Odom-R is the only other initialization technique which
assumes the same amount of prior information as SCORE, i.e., no information
beyond the robot's measurements. In comparing SCORE to Odom-R we observe that
the SCORE initialization typically obtains an average APE roughly two orders of
magnitude less than that obtained by Odom-R initialization. In fact, SCORE
appears to obtain more accurate trajectory estimates than those generated by
Odom-P, which assumes knowledge of each robot's first pose. These
observations suggest that SCORE is a more effective initialization strategy than
general odometry-based techniques (the current state-of-the-art initialization
for these problems) even when each robot's first pose is
available.

We note that several instances demonstrated substantial improvement in the
estimate by refining the SCORE initialization. In particular, we observed the
translation estimates were compressed (i.e., the estimated distance traveled was
notably less than the true value of 1 meter). Despite this compression in the
initialized translations, the final MAP estimate appeared high quality. We
hypothesize that good solutions are obtained from these compressed translations
because the SCORE initialization returns good initial estimates of the robot
orientations and beacon locations. This would agree with known results in
pose-graph SLAM \cite{carlone15icra} which, show that the quality of MAP
solutions depends more on good orientation initializations than good
translation initializations.


\subsection{When Does SCORE Return Poor Initializations?}
\label{sec:limitations}

\ScoreDeterminantPlot

Though SCORE generally performed well as an initialization technique, we
observed several multi-robot RA-SLAM instances where SCORE produced poor
initializations. These poor initializations were due to the cost function
tending towards a trivial solution of all zeros (i.e., $\rot_i = 0_{d \times d},
\tran_i = \dist_i = 0_{d}$). While the constraint $\rot_0 = I_{d}$ should
prevent this, in practice certain scenarios demonstrate the
remaining variables rapidly decaying to zero. This phenomenon is unsurprising,
as it is found in similar SLAM relaxations \cite{Rosen15icra,carlone15icra} and
the zeros elements indeed minimizes the problem cost. Importantly, we
note that these failures can be readily detected by observing the determinants
of the estimated rotations, which will quickly decay to less than
$\expnumber{5}{-2}$ (see \cref{fig:rotation-determinants}).
As a result, failures can be identified and alternative initializations used in
this case.

We find that the experiments with failed (poor) initializations resemble the
conditions identified by \cite{tseng07siam} for the SNL SOCP derived.  Namely,
the SOCP relaxation returns good solutions when the robot trajectories had
well-constrained poses (via range measurements) within the convex hull of robot
poses connected via odometry to $\rot_0$ (i.e., the trajectory of the robot with
constrained first rotation). Though at present we lack formal analysis,
empirical results suggest the analysis and connections to rigidity theory in
\cite{tseng07siam} are closely linked to SCORE.
